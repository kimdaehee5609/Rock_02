%	-------------------------------------------------------------------------------
% 
%
%
%
%
%
%
%
%
%
%	-------------------------------------------------------------------------------

	\documentclass[12pt, a4paper, oneside]{book}
%	\documentclass[12pt, a4paper, landscape, oneside]{book}

		% --------------------------------- 페이지 스타일 지정
		\usepackage{geometry}
%		\geometry{landscape=true	}
		\geometry{top 		=10em}
		\geometry{bottom		=10em}
		\geometry{left		=8em}
		\geometry{right		=8em}
		\geometry{headheight	=4em} % 머리말 설치 높이
		\geometry{headsep		=2em} % 머리말의 본문과의 띠우기 크기
		\geometry{footskip		=4em} % 꼬리말의 본문과의 띠우기 크기
% 		\geometry{showframe}
	
%		paperwidth 	= left + width + right (1)
%		paperheight 	= top + height + bottom (2)
%		width 		= textwidth (+ marginparsep + marginparwidth) (3)
%		height 		= textheight (+ headheight + headsep + footskip) (4)



		%	===================================================================
		%	package
		%	===================================================================
%			\usepackage[hangul]{kotex}				% 한글 사용
			\usepackage{kotex}						% 한글 사용
			\usepackage[unicode]{hyperref}			% 한글 하이퍼링크 사용
			\usepackage{amssymb,amsfonts,amsmath}	% 수학 수식 사용

			\usepackage{scrextend}					% 
		
			\usepackage{enumerate}			%
			\usepackage{enumitem}			%
			\usepackage{tablists}			%	수학문제의 보기 등을 표현하는데 사용
										%	tabenum


		% ------------------------------ table 
			\usepackage{longtable}			%
			\usepackage{tabularx}			%

			\usepackage{setspace}			%
			\usepackage{booktabs}			% table
			\usepackage{color}				%
			\usepackage{multirow}			%
			\usepackage{boxedminipage}		% 미니 페이지
			\usepackage[pdftex]{graphicx}	% 그림 사용
			\usepackage[final]{pdfpages}	% pdf 사용
			\usepackage{framed}			% pdf 사용
			
			\usepackage{fix-cm}	
			\usepackage[english]{babel}
	
			\usepackage{tikz}%
			\usetikzlibrary{arrows,positioning,shapes}
			%\usetikzlibrary{positioning}
			

		% --------------------------------- 	page
			\usepackage{afterpage}			% 다음페이지가 나온면 어떻게 하라는 명령 정의 패키지
%			\usepackage{fullpage}			% 잘못 사용하면 다 흐트러짐 주의해서 사용
%			\usepackage{pdflscape}			% 
			\usepackage{lscape}			%	 


			\usepackage{blindtext}
	
		% --------------------------------- font 사용
			\usepackage{pifont}				%
			\usepackage{textcomp}
			\usepackage{gensymb}
			\usepackage{marvosym}






		% --------------------------------- 페이지 스타일 지정

		\usepackage[Sonny]		{fncychap}

			\makeatletter
			\ChNameVar	{\Large\bf}
			\ChNumVar		{\Huge\bf}
			\ChTitleVar	{\Large\bf}
			\ChRuleWidth	{0.5pt}
			\makeatother

%		\usepackage[Lenny]		{fncychap}
%		\usepackage[Glenn]		{fncychap}
%		\usepackage[Conny]		{fncychap}
%		\usepackage[Rejne]		{fncychap}
%		\usepackage[Bjarne]	{fncychap}
%		\usepackage[Bjornstrup]{fncychap}

		\usepackage{fancyhdr}
		\pagestyle{fancy}
		\fancyhead{} % clear all fields
		\fancyhead[LO]{\footnotesize \leftmark}
		\fancyhead[RE]{\footnotesize \leftmark}
		\fancyfoot{} % clear all fields
		\fancyfoot[LE,RO]{\large \thepage}
		%\fancyfoot[CO,CE]{\empty}
		\renewcommand{\headrulewidth}{1.0pt}
		\renewcommand{\footrulewidth}{0.4pt}
	
	
	
		% --------------------------------- 	section 스타일 지정
	
		\usepackage{titlesec}
		
		\titleformat*{\section}			{\large\bfseries}
		\titleformat*{\subsection}			{\normalsize\bfseries}
		\titleformat*{\subsubsection}		{\normalsize\bfseries}
		\titleformat*{\paragraph}			{\normalsize\bfseries}
		\titleformat*{\subparagraph}		{\normalsize\bfseries}
	
		\renewcommand{\thesection}			{\arabic{section}.}
		\renewcommand{\thesubsection}		{\thesection\arabic{subsection}.}
		\renewcommand{\thesubsubsection}	{\thesubsection\arabic{subsubsection}}
		
		\titlespacing*{\section} 			{0pt}{1.0em}{1.0em}
		\titlespacing*{\subsection}	  		{0ex}{1.0em}{1.0em}
		\titlespacing*{\subsubsection}		{0ex}{1.0em}{1.0em}
		\titlespacing*{\paragraph}			{0ex}{1.0em}{1.0em}
		\titlespacing*{\subparagraph}		{0ex}{1.0em}{1.0em}
	
	%	\titlespacing*{\section} 			{0pt}{0.0\baselineskip}{0.0\baselineskip}
	%	\titlespacing*{\subsection}	  		{0ex}{0.0\baselineskip}{0.0\baselineskip}
	%	\titlespacing*{\subsubsection}		{6ex}{0.0\baselineskip}{0.0\baselineskip}
	%	\titlespacing*{\paragraph}			{6pt}{0.0\baselineskip}{0.0\baselineskip}
	

		% --------------------------------- recommend		섹션별 페이지 상단 여백
		\newcommand{\SectionMargin}			{\newpage  \null \vskip 2cm}
		\newcommand{\SubSectionMargin}		{\newpage  \null \vskip 2cm}
		\newcommand{\SubSubSectionMargin}	{\newpage  \null \vskip 2cm}


	
		% --------------------------------- 장의 목차
		\usepackage{minitoc}
		\setcounter{minitocdepth}{1}    	% Show until subsubsections in minitoc
		\setlength{\mtcindent}{12pt} 		% default 24pt
	
	
		% --------------------------------- 	문서 기본 사항 설정
		\setcounter{secnumdepth}{3} 		% 문단 번호 깊이
		\setcounter{tocdepth}{3} 			% 문단 번호 깊이
		\setlength{\parindent}{0cm} 		% 문서 들여 쓰기를 하지 않는다.
		
		
		% --------------------------------- 	줄간격 설정
		\doublespace
%		\onehalfspace
%		\singlespace
		
		
% 	============================================================================== List global setting
%		\setlist{itemsep=1.0em}
	
% 	============================================================================== enumi setting

%		\renewcommand{\labelenumi}{\arabic{enumi}.} 
%		\renewcommand{\labelenumii}{\arabic{enumi}.\arabic{enumii}}
%		\renewcommand{\labelenumii}{(\arabic{enumii})}
%		\renewcommand{\labelenumiii}{\arabic{enumiii})}


	%	-------------------------------------------------------------------------------
	%		Vertical and Horizontal spacing
	%	-------------------------------------------------------------------------------
		\setlist[enumerate,1]	{ leftmargin=8.0em, rightmargin=0.0em, labelwidth=0.0em, labelsep=0.0em }
		\setlist[enumerate,2]	{ leftmargin=8.0em, rightmargin=0.0em, labelwidth=0.0em, labelsep=0.0em }
		\setlist[enumerate,3]	{ leftmargin=8.0em, rightmargin=0.0em, labelwidth=0.0em, labelsep=0.0em }
		\setlist[enumerate]	{ 	itemsep=1.0em, 
								leftmargin=6.0ex, 
								rightmargin=0.0em, 
								labelwidth=0.0em, 
								labelsep=4.0ex 
							}


	%	-------------------------------------------------------------------------------
	%		Label
	%	-------------------------------------------------------------------------------
%		\setlist[enumerate,1]{ label=\arabic*., ref=\arabic* }
%		\setlist[enumerate,1]{ label=\emph{\arabic*.}, ref=\emph{\arabic*} }
%		\setlist[enumerate,1]{ label=\textbf{\arabic*.}, ref=\textbf{\arabic*} }   	% 1.
%		\setlist[enumerate,1]{ label=\textbf{\arabic*)}, ref=\textbf{\arabic*)} }		% 1)
		\setlist[enumerate,1]{ label=\textbf{(\arabic*)}, ref=\textbf{(\arabic*)} }	% (1)
		\setlist[enumerate,2]{ label=\textbf{\arabic*)}, ref=\textbf{\arabic*)} }		% 1)
		\setlist[enumerate,3]{ label=\textbf{\arabic*.}, ref=\textbf{\arabic*.} }		% 1.

%		\setlist[enumerate,2]{ label=\emph{\alph*}),ref=\theenumi.\emph{\alph*} }
%		\setlist[enumerate,3]{ label=\roman*), ref=\theenumii.\roman* }


% 	============================================================================== itemi setting


	%	-------------------------------------------------------------------------------
	%		Vertical and Horizontal spacing
	%	-------------------------------------------------------------------------------
		\setlist[itemize]{itemsep=0.0em}


	%	-------------------------------------------------------------------------------
	%		Label
	%	-------------------------------------------------------------------------------
		\renewcommand{\labelitemi}{$\bullet$}
		\renewcommand{\labelitemii}{$\cdot$}
		\renewcommand{\labelitemiii}{$\diamond$}
		\renewcommand{\labelitemiv}{$\ast$}		




		% --------------------------------- recommend  글자 색깔지정 명령
		\newcommand{\red}		{\color{red}}			% 글자 색깔 지정
		\newcommand{\blue}		{\color{blue}}		% 글자 색깔 지정
		\newcommand{\black}	{\color{black}}		% 글자 색깔 지정
		\newcommand{\superscript}[1]{${}^{#1}$}

	
	
		% --------------------------------- 환경 정의 : 박스 치고 안의 글자 빨간색

			\newenvironment{BoxRedText}
			{ 	\setlength{\fboxsep}{12pt}
				\begin{boxedminipage}[c]{1.0\linewidth}
				\color{red}
			}
			{ 	\end{boxedminipage} 
				\color{black}
			}
			
			

% ------------------------------------------------------------------------------
% Begin document (Content goes below)
% ------------------------------------------------------------------------------
	\begin{document}
	
			\dominitoc
			

			\title{변형작용과 지질구조}
			\author{김대희}
			\date{2017년 2월}
			\maketitle


			\tableofcontents
%			\listoffigures
%			\listoftables

			


	\clearpage
% ================================================= chapter 	====================
	\chapter{변형 작용}


	\clearpage
	% ------------------------------------------ section ------------ 
	\section{변형작용}



	

			\begin{itemize}[	topsep=0.0em, itemsep=0.0em, leftmargin=4em, labelsep=3em ] 
			\item	지각을 구성하고 있는 암반이 응력(stress)을 받으면 변형작용이 발생한다.
			\item	이 변형작용이 일어나는 곳의 \textbf{조건(온도 및 압력)}에 따라 	변형 양상이 다르게 나타난다.
			\item	온도 및 압력이 높은 지하 심부에서는 \textbf{연성변형작용}이 일어나 \textbf{연성전단대}나 \textbf{습곡}과 같은 지질구조가 생길 것이다. 
			\item	반면 온도가 낮은 지각의 천부에서는 암반이 깨어지는 \textbf{취성변형작용}이 일어나 
					\textbf{절리}나 \textbf{단층}과 같은 지질구조가 형성된다.
			\end{itemize}	



		\paragraph{지 구조 운동}




		\paragraph{심부의 연성변형}




		\paragraph{천부의 취성변형}





	\clearpage
% ================================================= chapter 	====================
	\chapter{연성전단대}

	\clearpage
	% ------------------------------------------ section ------------ 
	\section{연성전단대}


	\clearpage
	% ------------------------------------------ section ------------ 
	\section{순창전단대, 호남전단대}









	\clearpage
% ================================================= chapter 	====================
	\chapter{습곡}


	\clearpage
	% ------------------------------------------ section ------------ 
	\section{습곡}
	
		층리를 가지는 퇴적암이나 엽리를 가지는 변성암들이 횡압력을 받아 물결모양으로 굴곡된 형태를 가지는 지질구조를 `습곡'이라 한다.
	


	\clearpage
	% ------------------------------------------ section ------------ 
	\section{습곡의 각 부분 명칭}


	\subsection{배사}

	\subsection{향사}
	
	\subsection{힌지}


	\subsection{습곡축}
	
	\subsection{습곡의 날개}
	
	\subsection{습곡 축면}
	



	\subsection{배사구조}
	\subsection{향사구조}
	\subsection{단사구조}
	
	
	



	
	\subsection{횡와습곡}
	\subsection{경사습곡}
	
	\subsection{완사습곡}
	\subsection{급사습곡}
	\subsection{등사습곡}
	
	
	





	\clearpage
	% ------------------------------------------ section ------------ 
	\section{습곡의 종류}



	\subsection{성인에 의한 분류}

	\subsection{모양에 의한 분류}


	
	\clearpage
% ================================================= chapter 	====================
	\chapter{절리}



	\clearpage
	% ------------------------------------------ section ------------ 
	\section{절리}

	\subsection{단열 fracture}


	\paragraph{절리}
	절리는 암석의 갈라진 틈으로서 절리면을 중심으로 양쪽 암체의 상대적 변위가 없는 것을 말한다.
	
	\paragraph{전단 절리}
	
	\paragraph{인장 절리}
	


	\paragraph{깃털 구조}
	
	
	\paragraph{판상 절리}


	
	\paragraph{주상 절리}
	
	





	\clearpage
	% ------------------------------------------ section ------------ 
	\section{절리의 종류 및 분류}
	
	\subsection{기하학적 분류}
	
	\subsection{형태적 분류}
	
	\subsection{성인별 분류}
	
	
	
	
	
	
	\clearpage
% ================================================= chapter 	====================
	\chapter{단층}


	\clearpage
	% ------------------------------------------ section ------------ 
	\section{단층}

		\paragraph{단층}
		지각 내에 작용한 응력에 의해ㅔ 취성변형작용이  일어나면 단열대가 형성된다.
		이 단열대를 중심으로 변위가 발생하여 양 블록에 상대적으로 이동한 것이 단층이다.
	
		\paragraph{단층 활면}
	
	
		\paragraph{단층 조선}
	
	
		\paragraph{단층 점토}
		단층이 미끄러질 때에 암석이 갈려 점토화 된것
			
	
		\paragraph{단층 각력}
		각력으로 간극에 남아 있는 것을 단층각력
		
		\paragraph{압쇄암}
		
	
		\paragraph{단층대}

		\paragraph{단층 파쇄대}




	\clearpage
	% ------------------------------------------ section ------------ 
	\section{단층의 종류}




	\clearpage
	% ------------------------------------------ section ------------ 
	\section{정단층}
	
	
	\clearpage
	% ------------------------------------------ section ------------ 
	\section{역단층}
	
	\clearpage
	% ------------------------------------------ section ------------ 
	\section{주향이동 단층}
	
		\paragraph{주향이동 단층}
		

		\paragraph{우수향 주향이동 단층}


		\paragraph{좌수향 주향이동 단층}






	\clearpage
	% ------------------------------------------ section ------------ 
	\section{사교단층}
	


	\clearpage
	% ------------------------------------------ section ------------ 
	\section{회전단층}
	


	\clearpage
	% ------------------------------------------ section ------------ 
	\section{스러스트 단층}


	\clearpage
	% ------------------------------------------ section ------------ 
	\section{단층의 구분 방법}
	
	
	\clearpage
	% ------------------------------------------ section ------------ 
	\section{단층조사법과 건설공사}
	
	
	
	
		\paragraph{건설공사에서의 단층파쇄대의 문제점}
	
	
		\paragraph{터널공사에서의 단층파쇄대의 문제점}
	
		\paragraph{경사면에서의 단층파쇄대의 문제점}
	
		\paragraph{댐공사에서의 단층파쇄대의 문제점}
	



	\clearpage
	% ------------------------------------------ section ------------ 
	\section{활성단층의 정의}
	
		\paragraph{활성단층의 정의}



	\clearpage
	% ------------------------------------------ section ------------ 
	\section{단층의 등급분류}
	
	



	
	\clearpage
% ================================================= chapter 	====================
	\chapter{부정합}
	

	\clearpage
	% ------------------------------------------ section ------------ 
	\section{부정합}
	
	
	\clearpage
	% ------------------------------------------ section ------------ 
	\section{부정합의 종류}
	
	
	
	\clearpage
	% ------------------------------------------ section ------------ 
	\section{평행부정합과 비정합}



	\clearpage
	% ------------------------------------------ section ------------ 
	\section{난정합}
	
	
	\clearpage
	% ------------------------------------------ section ------------ 
	\section{경사부정합과 사교부정합}
	
	
	\clearpage
	% ------------------------------------------ section ------------ 
	\section{부정합의 인지}
	
	
	
		\paragraph{노두 관찰}
		
		
		\paragraph{지질도 작성}
		
		
		\paragraph{변성정도와 습곡정도의 차이}
		


		\paragraph{고생물학적 조사}
		
		
	
	
	
	




	
	
	
	\clearpage
% ================================================= chapter 	====================
	\chapter{불연속면}
	
	
	\clearpage
	% ------------------------------------------ section ------------ 
	\section{불연속면}
	
	
	
		\paragraph{절리}
		
		\paragraph{층리면}
		
		\paragraph{단층}
	

		\paragraph{파쇄대}
		
		\paragraph{벽개}
		
		\paragraph{편리}
		
		\paragraph{단열}
		
	

	\clearpage
	% ------------------------------------------ section ------------ 
	\section{불연속면의 성질 중 중요한 요소}

		\paragraph{불연속면의 방향성}
		
		\paragraph{간격}
		\paragraph{연속성}
		\paragraph{거칠기}
		\paragraph{벽면강도}
		
		\paragraph{간극}
		
		\paragraph{충전물}
		
		\paragraph{누수}
		
		\paragraph{불연속면의 수}
		
		\paragraph{암괴의 크기}
		
		
	
	
	
	\clearpage
	% ------------------------------------------ section ------------ 
	\section{방향성}
	
	
	
	\clearpage
	% ------------------------------------------ section ------------ 
	\section{간격}
	
	
	\clearpage
	% ------------------------------------------ section ------------ 
	\section{연속성}
	
	
	
	
	\clearpage
	% ------------------------------------------ section ------------ 
	\section{거칠기}
	
	
	
	\clearpage
	% ------------------------------------------ section ------------ 
	\section{벽면강도}
	
	
	\clearpage
	% ------------------------------------------ section ------------ 
	\section{간극}
	
	
	
	
	\clearpage
	% ------------------------------------------ section ------------ 
	\section{충전물}
	
	
	\clearpage
	% ------------------------------------------ section ------------ 
	\section{누수}
	
	
	
	\clearpage
	% ------------------------------------------ section ------------ 
	\section{불연속면군의 수}
	
	
	
	\clearpage
	% ------------------------------------------ section ------------ 
	\section{암괴의 크기 및 모양}
	
	
	
	
	
	
	
	
	
	
	
	
	
	
	\clearpage
% ================================================= chapter 	====================
	\chapter{취성파괴이론}



	
	\clearpage
	% ------------------------------------------ section ------------ 
	\section{취성파괴이론}
	

	
	
	
	
	



	

% ------------------------------------------------------------------------------
% End document
% ------------------------------------------------------------------------------
\end{document}


